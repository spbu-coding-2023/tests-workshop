\documentclass{beamer}
%Для защит онлайн лучше использовать разрешение 16x9
%\documentclass[aspectratio=169]{beamer}

\input{preamble.tex}

% То, что в квадратных скобках, отображается внизу по центру каждого слайда.
\title[Тестирование]{Тестирование и Kotlin}

% То, что в квадратных скобках, отображается в левом нижнем углу.
\institute[СПбГУ]{}

% То, что в квадратных скобках, отображается в левом нижнем углу.
\author[Кутуев Владимир]{}

\begin{document}
{
\setbeamertemplate{footline}{}
% Лого университета или организации, отображается в шапке титульного листа
\begin{frame}
  \includegraphics[width=1.4cm]{pictures/SPbGU_Logo.png}
\vspace{-35pt}
\hspace{-10pt}
\begin{center}
   \begin{tabular}{c}
        \scriptsize{Санкт-Петербургский государственный университет} \\
        \scriptsize{Кафедра системного программирования}
    \end{tabular}
\titlepage
\end{center}

\btVFill

\begin{center}
  \vspace{5pt}
  \scriptsize{Санкт-Петербург\\
                 2024}
  \end{center}

\end{frame}
}

\begin{frame}[fragile]
  \frametitle{Аксиомы Шуры-Буры}
  \begin{enumerate}
    \item Любая программа содержит ошибки
    \item Если программа не содержит ошибок, то их содержит спецификация, по которой программа разрабатывалась
    \item Если ни программа, ни спецификация ошибок не содержат, то такая программа никому не нужна
  \end{enumerate}
\end{frame}

\begin{frame}
  \frametitle{Много ли ошибок в коде?}
  \begin{itemize}
    \item Согласно C. Kaner <<Testing computer software>> (1988)
    \begin{itemize}
      \item после передачи в тестирование 1-3 ошибки на 100 строк
      \item в процессе разработки 1.5 ошибки на 1 строку
      \item исправление 10 операторов происходит верно в 50\% случаев
    \end{itemize}
    \item Числительные не столь принципиальны, важна картина в целом...
  \end{itemize}

\end{frame}

\begin{frame}
  \frametitle{Тестирование}
  \begin{itemize}
    \item Система --- белый ящик
    \item Система --- чёрный ящик
  \end{itemize}
\end{frame}

\begin{frame}
  \frametitle{Ручное тестирование}
  \begin{itemize}
    \item Наиболее дешёвое
    \item Требует несколько меньшей квалификации
    \item Тестирование UI сложно автоматизируется
    \item Можно быстро произвести в ходе разработки
  \end{itemize}
\end{frame}

\begin{frame}
  \frametitle{Автоматизированное тестирование}
  \begin{itemize}
    \item Требует навыков программирования
    \item Не любой код можно написать автотестируемым
    \item Тесты требуется поддерживать и сопровождать
    \item Чем чаще релизы, тем оно нужнее
  \end{itemize}
\end{frame}

\begin{frame}
  \frametitle{Модульные (Unit) тесты}
  \begin{itemize}
    \item Тестирование отдельного класса
    \item Проверяют внешнее поведение класса
    \item Полностью автоматические
    \item Направлены на поиск ошибок в конкретном методе
    \item Не влияют на функциональность системы и не поставляются пользователю
  \end{itemize}
\end{frame}

\begin{frame}
  \frametitle{Три А}
  \begin{itemize}
    \item Arrange
    \item Act
    \item Assert
  \end{itemize}
\end{frame}

\begin{frame}[fragile]
  \frametitle{Пример 1}
  \begin{Verbatim}[commandchars=\\\{\}]
\textcolor{blue}{public fun} max(x: \textcolor{blue}{Int}, y: \textcolor{blue}{Int}): \textcolor{blue}{Int} \{
    \textcolor{purple}{if} (x > y) \{
        \textcolor{blue}{return} x;
    \} \textcolor{purple}{else} \{
        \textcolor{blue}{return} y;
    \}
\}
  \end{Verbatim}
\end{frame}

\begin{frame}[fragile]
  \frametitle{Пример 2}
  \begin{Verbatim}[commandchars=\\\{\}]
\textcolor{blue}{public fun} simpleCondition(c: \textcolor{blue}{SimpleDataClass}): \textcolor{blue}{Int} \{
    \textcolor{purple}{if} (c.a < 5 && c.b > 10) \{
        \textcolor{blue}{return} 0;
    \} \textcolor{purple}{else} \{
        c.a = 3;
        \textcolor{blue}{return} c.a;
    \}
\}
  \end{Verbatim}
\end{frame}

\begin{frame}[fragile]
  \frametitle{Пример 3}
  \begin{Verbatim}[commandchars=\\\{\}]
\textcolor{blue}{public fun} byteArray(a: \textcolor{blue}{Array<Byte>}, x: \textcolor{blue}{Byte}): \textcolor{blue}{Byte} \{
    \textcolor{purple}{if} (a.length != 2) \{
        \textcolor{blue}{return} -1;
    \}
    a[0] = 5;
    a[1] = x;
    \textcolor{purple}{if} (a[0] + a[1] > 20) \{
        \textcolor{blue}{return} 1;
    \}
    \textcolor{blue}{return} 0;
\}
  \end{Verbatim}
\end{frame}

\begin{frame}
  \frametitle{Kotlin}
  \begin{itemize}
    \item JUnit 4, JUnit 5, TestNg --- наследие Java
    \item AssertJ, AssertK
    \item Mocking: Mockito (Mockito-Kotlin), Powermock
    \item BDD Tests: kotest, spek
  \end{itemize}
\end{frame}

\begin{frame}
  \frametitle{Best practices (Для Unit-тестов)}
  \begin{itemize}
    \item Независимость тестов
    \begin{itemize}
      \item Желательно, чтобы поломка одного куска функциональности ломала один тест
    \end{itemize}
    \item Тесты должны работать быстро
    \begin{itemize}
      \item И запускаться после каждой сборки (CI)
    \end{itemize}
    \item Тестов должно быть много
    \begin{itemize}
      \item Следить за Code coverage
    \end{itemize}
    \item Каждый тест должен проверять конкретный тестовый сценарий
    \item Test-driven development
  \end{itemize}
\end{frame}

\begin{frame}
  \frametitle{Test driven development (TDD)}
  Разработка непрерывно проходит следующие этапы:
  \begin{itemize}
    \item идея новой функциональности
    \item написание теста: как оно должно работать
    \item проверка, что тесты падают
    \item первичная реализация
    \item отладка до прохождения теста
    \item проверка прохождения предыдущих тестов
    \item рефакторинг
    \item проверка прохождения теста
  \end{itemize}
\end{frame}

\end{document}
